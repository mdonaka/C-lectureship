%%%%%%%%%%%%%%%%%%%%%%%%%%%%%%%%%%%%%%%%%%%%%%%%%%%%%%

\part{ジェネリックプログラミング}
ジェネリックプログラミングとは特定のデータ型に依存せず,
抽象的で汎用的な実装を行うプログラミング手法である.
ジェネリックを効果的に用いることで,
データ型を気にせず自由にかつ安全なコーディングが出来る.
さらに,C++の代表的なライブラリであるSTL(Standard Template Library)は
ジェネリックの思想で構成されているため,
この概念を理解することでよりSTLを上手に利用できるようになる.

本章ではC++ジェネリックプログラミングの基本となるテンプレートを中心に,
型推論やジェネリックラムダ等について学んでいく.

%%%%%%%%%%%%%%%%%%%%%%%%%%%%%%%%%%%%%%%%%%%%%%%%%%%%%%
%%%%%%%%%%%%%%%%%%%%%%%%%%%%%%%%%%%%%%%%%%%%%%%%%%%%%%
\section{auto(C++11)}
autoは型推論を行うためのキーワードであり,
型の代わりに用いることで初期化子から型を推論する.
実際の使用例を\cdref{cd:gen01}に示す.
\lstinputlisting[caption=example of auto keyword,label=cd:gen01]{code/generic/01.cpp}

%%%%%%%%%%%%%%%%%%%%%%%%%%%%%%%%%%%%%%%%%%%%%%%%%%%%%%
\subsection{autoの効果的な使いどころ}
autoは非常に便利であるが,特に型名が長い場合等で絶大な効果を発揮する.
例えば,for文でコンテナのイテレータを回す際はイテレータを変数にキャッチするが,その長い型を4字で記述することが出来る.
さらに,\scref{sc:lambda}で説明するラムダ式を宣言する場合でも型を知る必要がなくなる.
\lstinputlisting[caption=effective usage of auto,label=cd:gen02]{code/generic/02.cpp}

なお補足程度ではあるが,\cdref{cd:gen03}に示すようにラムダ式は実装が同じであっても型が異なる.
これはラムダ式が関数オブジェクトの一種であり宣言の度に異なるクラスのオブジェクトとして生成されるためである.
このように宣言の度に変化する型を厳密に定義することは不可能のため,
autoの重要さが際立つ.
\lstinputlisting[caption=different type of lambda,label=cd:gen03]{code/generic/03.cpp}

%%%%%%%%%%%%%%%%%%%%%%%%%%%%%%%%%%%%%%%%%%%%%%%%%%%%%%
\subsection{修飾を伴うauto}
autoが推論するのは初期化子の型ではなく,
あくまで代入するのに最低限必要な型ということに注意する必要がある.
\cdref{cd:gen04}で示すように,
constや参照が付与された型をそのまま推論すると
推論された型もconstや参照が付与されるはずである.
しかしながらautoにより推論された型はそれらの修飾が外れる.

constや参照を付与した型宣言をしたい場合は,
\cdref{cd:gen05}のようにautoに対して修飾することで実現できる.
\lstinputlisting[caption=auto keyword with qualification,label=cd:gen04]{code/generic/04.cpp}
\lstinputlisting[caption=add qualification to auto keyword,label=cd:gen05]{code/generic/05.cpp}

%%%%%%%%%%%%%%%%%%%%%%%%%%%%%%%%%%%%%%%%%%%%%%%%%%%%%%
\subsection{initializer lists(C++11)}
initializer listsは最も抽象化されたコンテナのようなものであり,
$\{1,2,3\}$のような列を型推論した場合に発生する.
vector等のコンテナ型の初期化をautoと$\{\textrm{hoge},\textrm{huga}\}$を用いてする場合,
コンテナを明示しなければ型がinitializer listsと判定されてしまうため注意が必要である.
\lstinputlisting[caption=feature of initializer lists,label=cd:gen06]{code/generic/06.cpp}

%%%%%%%%%%%%%%%%%%%%%%%%%%%%%%%%%%%%%%%%%%%%%%%%%%%%%%
\subsection{関数の戻り値型推論(C++14)}
autoを使うことで関数の戻り値も型推論で完結させることができる.
使用例を\cdref{cd:gen10}に示す.
\lstinputlisting[caption=function return type inference,label=cd:gen10]{code/generic/10.cpp}

%%%%%%%%%%%%%%%%%%%%%%%%%%%%%%%%%%%%%%%%%%%%%%%%%%%%%%

%%%%%%%%%%%%%%%%%%%%%%%%%%%%%%%%%%%%%%%%%%%%%%%%%%%%%%
\section{decltype(C++11)}
decltypeは引数として受け取った値の型を返す指定子である.
autoのように初期化子から型を推論して使うだけでなく,
いつでも任意の変数を型を推論して使用することができる.
実際の使用例を\cdref{cd:gen07}に示す.
\lstinputlisting[caption=example of decltype specifier,label=cd:gen07]{code/generic/07.cpp}

%%%%%%%%%%%%%%%%%%%%%%%%%%%%%%%%%%%%%%%%%%%%%%%%%%%%%%
\subsection{修飾を伴うdecltype}
decltypeはautoとは異なり,修飾も含めて全ての型を推論する.
autoのように使用して値をコピーしたつもりだったが,
参照を渡してしまってバグが起きた,
等が考えられるので使用する場合は修飾も含めてよく確認するようにする.
\lstinputlisting[caption=decltype specifier with qualification,label=cd:gen08]{code/generic/08.cpp}
%%%%%%%%%%%%%%%%%%%%%%%%%%%%%%%%%%%%%%%%%%%%%%%%%%%%%%
\subsection{修飾外し}
constで受け取った引数をdelctypeで推論する,
参照で受け取った引数をdelctypeで推論するといったことはジェネリックプログラミングではよくあることである.
しかしながら,ほとんどのケースではconstや参照が邪魔となり上手く機能しない.
そこで,C++にはこういった修飾を外すためのクラスが容易されている.
\lstinputlisting[caption=Reference and const deletion,label=cd:gen09]{code/generic/09.cpp}
%%%%%%%%%%%%%%%%%%%%%%%%%%%%%%%%%%%%%%%%%%%%%%%%%%%%%%
\subsection{decltype(auto)(C++14)}
値のコピーを初期化子の型に合わせて行う場合,
autoやdecltype(初期化子)とする方法があるが,
C++14ではdecltype(初期化子)をdecltype(auto)と記載できるようになった.
これにより,参照やconstを維持したコピーを完結に記載できる.
\lstinputlisting[caption=example of decltype(auto),label=cd:gen11]{code/generic/11.cpp}
%%%%%%%%%%%%%%%%%%%%%%%%%%%%%%%%%%%%%%%%%%%%%%%%%%%%%%
%%%%%%%%%%%%%%%%%%%%%%%%%%%%%%%%%%%%%%%%%%%%%%%%%%%%%%
%%%%%%%%%%%%%%%%%%%%%%%%%%%%%%%%%%%%%%%%%%%%%%%%%%%%%%

%%%%%%%%%%%%%%%%%%%%%%%%%%%%%%%%%%%%%%%%%%%%%%%%%%%%%%
\section{テンプレート}
テンプレートはデータ型を抽象化して定義することにより,
あらゆる型の動作を定義するための仕組みである.
これを用いることで特定の型に依存しない汎用的な関数やクラス等を記述できる.
%%%%%%%%%%%%%%%%%%%%%%%%%%%%%%%%%%%%%%%%%%%%%%%%%%%%%%
\subsection{基本のテンプレート}
テンプレートは以下のように宣言を行い,
この宣言後に関数やクラスを宣言することでテンプレート関数,テンプレートクラス等を生成できる.
\begin{lstlisting}[]
template <class 型名>
\end{lstlisting}

テンプレートの基本的な使い方は大きく4つある.
本節では4つのテンプレートの基本的な使い方を記述する.

%%%%%%%%%%%%%%%%%%%%%%%%%%%%%%%%%%%%%%%%%%%%%%%%%%%%%%]
\subsubsection*{関数テンプレート}
関数をテンプレート化したのが関数テンプレートである.
テンプレート宣言後に関数を宣言することで生成できる.
関数テンプレートでは返り値や引数,
ローカル変数等様々な位置で抽象化された型を使用できる.

\cdref{cd:gen12}では,値を二つ受け取りそれらの最大値を返す関数を定義している.
返り値と引数の型を抽象化することで,あらゆる値に対して有効な関数が生成出来ている.
\lstinputlisting[caption=example of template function,label=cd:gen12]{code/generic/12.cpp}

%%%%%%%%%%%%%%%%%%%%%%%%%%%%%%%%%%%%%%%%%%%%%%%%%%%%%%]
\subsubsection*{クラステンプレート}
クラスをテンプレート化したのがクラステンプレートである.
テンプレート宣言後にクラスを宣言することで生成できる.
クラステンプレートでは内部のあらゆる値の型で抽象化された型を使用できる.

\lstinputlisting[caption=example of template class,label=cd:gen13]{code/generic/13.cpp}
%%%%%%%%%%%%%%%%%%%%%%%%%%%%%%%%%%%%%%%%%%%%%%%%%%%%%%
\subsubsection*{メンバテンプレート}
メンバ関数をテンプレート化したのがメンバテンプレートである.
テンプレート宣言後にメンバを宣言することで生成できる.
クラス内の一部のメソッドだけをテンプレート化したいという場合で用いられる.
またクラステンプレートとの共存も可能で,2重のテンプレートを持つメンバも定義できる.

\lstinputlisting[caption=example of template member function,label=cd:gen14]{code/generic/14.cpp}
%%%%%%%%%%%%%%%%%%%%%%%%%%%%%%%%%%%%%%%%%%%%%%%%%%%%%%
\subsubsection*{エイリアステンプレート}
エイリアステンプレートは型の別名定義をテンプレートを用いて行うものである.
エイリアステンプレートはテンプレートクラスの型名を短く再定義する等
の使い道がある.
\lstinputlisting[caption=example of alias templates,label=cd:gen15]{code/generic/15.cpp}

補足だが,エイリアステンプレートは特殊化が出来ないことを利用して,
\cdref{cd:gen16}のようにユーザにエイリアスのみを提示して特殊化を防ぐテクニックがある.
他にも,真偽値メタ関数の論理演算を簡潔に定義したりメタ関数をラップしたりする用途がある.
実際に使用される場合はこれらの用途が主である.
\lstinputlisting[caption=using alias templates to prevent specialization,label=cd:gen16]{code/generic/16.cpp}
%%%%%%%%%%%%%%%%%%%%%%%%%%%%%%%%%%%%%%%%%%%%%%%%%%%%%%
\subsection{オーバーロード}
名前が同じで引数や戻り値が異なる関数やメソッド等を定義することをオーバーロードと呼ぶ.
オーバーロードはコンパイル時に全て解決するため,
実行中の関数切り替えのようなことはできない.
\lstinputlisting[caption=various overloads,label=cd:gen17]{code/generic/17.cpp}

%%%%%%%%%%%%%%%%%%%%%%%%%%%%%%%%%%%%%%%%%%%%%%%%%%%%%%
\subsubsection{特殊化}

\lstinputlisting[caption=example of template Specialization,label=cd:gen20]{code/generic/18.cpp}

%%%%%%%%%%%%%%%%%%%%%%%%%%%%%%%%%%%%%%%%%%%%%%%%%%%%%%
\subsubsection{部分特殊化}
\lstinputlisting[caption=example of partial template specialization,label=cd:gen20]{code/generic/19.cpp}

%%%%%%%%%%%%%%%%%%%%%%%%%%%%%%%%%%%%%%%%%%%%%%%%%%%%%%
\subsubsection{オーバーロードの優先度}
テンプレート等を駆使するようになるとオーバーロードをしていても複数の呼び出し候補が存在してしまうことがある.
その場合は以下の優先度で優先度の高い関数が呼ばれる.
例を\cdref{cd:gen21}に示す.
\begin{enumerate}
    \item 通常の呼び出し(修飾込)
    \item 特殊化テンプレート
    \item テンプレート
    \item 暗黙的な型変換で呼び出し可能なもの
    \item 可変長引数
\end{enumerate}

ここで注意すべき点として,同じ優先度内でオーバーロードの解決ができない場合はコンパイルエラーとなる.
例を\cdref{cd:gen20}に示す.
オーバーロードを記述するときは異なる優先度で解決できないか,
優先度内で衝突が起きてないか,等をしっかり確認する.
\lstinputlisting[caption=overload order,label=cd:gen21]{code/generic/21.cpp}
\lstinputlisting[caption=overload collision,label=cd:gen20]{code/generic/20.cpp}

%%%%%%%%%%%%%%%%%%%%%%%%%%%%%%%%%%%%%%%%%%%%%%%%%%%%%%
\section{SFINAE(C++11)}
SFINAE(Substitution Failure Is Not An Error)


\lstinputlisting[caption=1,label=cd:sfi1]{code/generic/sfinae/1.cpp}
\lstinputlisting[caption=1,label=cd:sfi1]{code/generic/sfinae/3.cpp}
\lstinputlisting[caption=1,label=cd:sfi1]{code/generic/sfinae/2.cpp}

%%%%%%%%%%%%%%%%%%%%%%%%%%%%%%%%%%%%%%%%%%%%%%%%%%%%%%
\subsection{メタ関数}

%%%%%%%%%%%%%%%%%%%%%%%%%%%%%%%%%%%%%%%%%%%%%%%%%%%%%%
\subsection{enable\_if}

%%%%%%%%%%%%%%%%%%%%%%%%%%%%%%%%%%%%%%%%%%%%%%%%%%%%%%
\subsection{自作メタ関数}
\lstinputlisting[caption=1,label=cd:sfi1]{code/generic/sfinae/4.cpp}

%%%%%%%%%%%%%%%%%%%%%%%%%%%%%%%%%%%%%%%%%%%%%%%%%%%%%%
\subsubsection{True,False type}

%%%%%%%%%%%%%%%%%%%%%%%%%%%%%%%%%%%%%%%%%%%%%%%%%%%%%%

%%%%%%%%%%%%%%%%%%%%%%%%%%%%%%%%%%%%%%%%%%%%%%%%%%%%%%
%%%%%%%%%%%%%%%%%%%%%%%%%%%%%%%%%%%%%%%%%%%%%%%%%%%%%%
%%%%%%%%%%%%%%%%%%%%%%%%%%%%%%%%%%%%%%%%%%%%%%%%%%%%%%
%%%%%%%%%%%%%%%%%%%%%%%%%%%%%%%%%%%%%%%%%%%%%%%%%%%%%%

%%%%%%%%%%%%%%%%%%%%%%%%%%%%%%%%%%%%%%%%%%%%%%%%%%%%%%
\section{ラムダ(C++11)}
\label{sc:lambda}

%%%%%%%%%%%%%%%%%%%%%%%%%%%%%%%%%%%%%%%%%%%%%%%%%%%%%%
\subsection{ジェネリックラムダ(C++14)}

%%%%%%%%%%%%%%%%%%%%%%%%%%%%%%%%%%%%%%%%%%%%%%%%%%%%%%
