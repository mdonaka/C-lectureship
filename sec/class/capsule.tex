%%%%%%%%%%%%%%%%%%%%%%%%%%%%%%%%%%%%%%%%%%%%%%%%%%%%%%
\section{カプセル化}
カプセル化はオブジェクト指向3大要素の一つであり,
真髄を理解するのになくてはならない存在である.
私はカプセル化について真に迫る概念を理解しているが,
ここに記すには余白が狭すぎる.
そこで本章ではその一端として,
クラス内methodがメンバ変数に及ぼす影響を減らす手法を紹介する.

%%%%%%%%%%%%%%%%%%%%%%%%%%%%%%%%%%%%%%%%%%%%%%%%%%%%%%

\subsection{static method}
\lstinputlisting[caption=wasteful class,label=cd:cl12]{code/class/12.cpp}
\lstinputlisting[caption=improved class,label=cd:cl13]{code/class/13.cpp}

%%%%%%%%%%%%%%%%%%%%%%%%%%%%%%%%%%%%%%%%%%%%%%%%%%%%%%

\subsection{namespace}
\lstinputlisting[caption=improved class with namespace,label=cd:cl14]{code/class/14.cpp}

%%%%%%%%%%%%%%%%%%%%%%%%%%%%%%%%%%%%%%%%%%%%%%%%%%%%%%
