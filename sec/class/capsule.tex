%%%%%%%%%%%%%%%%%%%%%%%%%%%%%%%%%%%%%%%%%%%%%%%%%%%%%%
\section{カプセル化}
カプセル化はオブジェクト指向3大要素の一つであり,
真髄を理解するのになくてはならない存在である.
私はカプセル化について真に迫る概念を理解しているが,
ここに記すには余白が狭すぎる.
そこで本章ではその一端として,
クラス内methodがメンバ変数に及ぼす影響を減らす手法を紹介する.

%%%%%%%%%%%%%%%%%%%%%%%%%%%%%%%%%%%%%%%%%%%%%%%%%%%%%%

\subsection{static method}
\cdref{cd:cl12}を見ると,methodという名前のmethodが宣言されている.
これはクラス内メンバ関数のため,インスタンス化される度に生成される.
しかしながら,methodはメンバ変数を参照していないため,全てのインスタンスで同じ挙動を起こす.
つまり,インスタンスの個数だけmethodが生成されるのは非効率的である.
そこで,\cdref{cd:cl13}のようにmethodをstatic宣言することで,
無駄なmethodの生成を防ぐテクニックがある.
\lstinputlisting[caption=wasteful class,label=cd:cl12]{code/class/12.cpp}
\lstinputlisting[caption=improved class,label=cd:cl13]{code/class/13.cpp}

%%%%%%%%%%%%%%%%%%%%%%%%%%%%%%%%%%%%%%%%%%%%%%%%%%%%%%

\subsection{namespace}
前節の\cdref{cd:cl14}は無駄な生成を防いではいるものの,
使用しないはずのメンバ変数をstatic methodが参照できてしまうため,
実はカプセル化がそこまで強くはない.
そこで,メンバ変数を一切使用しないmethodをクラスの外で定義するというテクニックを用いる.
そうすることで,メンバ変数に干渉できるmethodが減ってカプセル化がより強化される.
ただし,このテクニックを用いると関数がグローバル空間で宣言されてしまうため,
\cdref{cd:cl14}のようにnamespaceを用いてスコープを制御する必要がある.
\lstinputlisting[caption=improved class with namespace,label=cd:cl14]{code/class/14.cpp}


%%%%%%%%%%%%%%%%%%%%%%%%%%%%%%%%%%%%%%%%%%%%%%%%%%%%%%
