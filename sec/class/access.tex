%%%%%%%%%%%%%%%%%%%%%%%%%%%%%%%%%%%%%%%%%%%%%%%%%%%%%%

\section{アクセス修飾子}
C++でクラスは"struct","class"の2通りの宣言方法がある.
これらの唯一の違いはdefaultのアクセス修飾がprivateであるか,
publicであるかである.
\cdref{cd:cl01}に示すようにstructで宣言されたクラス内では
変数がpublic宣言となっているが,
classで宣言されたクラス内ではprivate宣言となっている.

どちらが良いということはないが,プロジェクト内でどちらを使うか統一する,役割に応じて使い分ける等すると良いだろう.

\lstinputlisting[caption=Implicit access qualification,label=cd:cl01]{code/class/01.cpp}

%%%%%%%%%%%%%%%%%%%%%%%%%%%%%%%%%%%%%%%%%%%%%%%%%%%%%%
