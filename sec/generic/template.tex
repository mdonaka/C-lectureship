%%%%%%%%%%%%%%%%%%%%%%%%%%%%%%%%%%%%%%%%%%%%%%%%%%%%%%
\section{テンプレート}
テンプレートはデータ型を抽象化して定義することにより,
あらゆる型の動作を定義するための仕組みである.
これを用いることで特定の型に依存しない汎用的な関数やクラス等を記述できる.
%%%%%%%%%%%%%%%%%%%%%%%%%%%%%%%%%%%%%%%%%%%%%%%%%%%%%%
\subsection{基本のテンプレート}
テンプレートは以下のように宣言を行い,
この宣言後に関数やクラスを宣言することでテンプレート関数,テンプレートクラス等を生成できる.
\begin{lstlisting}[]
template <class 型名>
\end{lstlisting}

テンプレートの基本的な使い方は大きく4つある.
本節では4つのテンプレートの基本的な使い方を記述する.

%%%%%%%%%%%%%%%%%%%%%%%%%%%%%%%%%%%%%%%%%%%%%%%%%%%%%%]
\subsubsection*{関数テンプレート}
関数をテンプレート化したのが関数テンプレートである.
テンプレート宣言後に関数を宣言することで生成できる.
関数テンプレートでは返り値や引数,
ローカル変数等様々な位置で抽象化された型を使用できる.

\cdref{cd:gen12}では,値を二つ受け取りそれらの最大値を返す関数を定義している.
返り値と引数の型を抽象化することで,あらゆる値に対して有効な関数が生成出来ている.
\lstinputlisting[caption=example of template function,label=cd:gen12]{code/generic/12.cpp}

%%%%%%%%%%%%%%%%%%%%%%%%%%%%%%%%%%%%%%%%%%%%%%%%%%%%%%]
\subsubsection*{クラステンプレート}
クラスをテンプレート化したのがクラステンプレートである.
テンプレート宣言後にクラスを宣言することで生成できる.
クラステンプレートでは内部のあらゆる値の型で抽象化された型を使用できる.

\lstinputlisting[caption=example of template class,label=cd:gen13]{code/generic/13.cpp}
%%%%%%%%%%%%%%%%%%%%%%%%%%%%%%%%%%%%%%%%%%%%%%%%%%%%%%
\subsubsection*{メンバテンプレート}
メンバ関数をテンプレート化したのがメンバテンプレートである.
テンプレート宣言後にメンバを宣言することで生成できる.
クラス内の一部のメソッドだけをテンプレート化したいという場合で用いられる.
またクラステンプレートとの共存も可能で,2重のテンプレートを持つメンバも定義できる.

\lstinputlisting[caption=example of template member function,label=cd:gen14]{code/generic/14.cpp}
%%%%%%%%%%%%%%%%%%%%%%%%%%%%%%%%%%%%%%%%%%%%%%%%%%%%%%
\subsubsection*{エイリアステンプレート}
エイリアステンプレートは型の別名定義をテンプレートを用いて行うものである.
エイリアステンプレートはテンプレートクラスの型名を短く再定義する等
の使い道がある.
\lstinputlisting[caption=example of alias templates,label=cd:gen15]{code/generic/15.cpp}

補足だが,エイリアステンプレートは特殊化が出来ないことを利用して,
\cdref{cd:gen16}のようにユーザにエイリアスのみを提示して特殊化を防ぐテクニックがある.
他にも,真偽値メタ関数の論理演算を簡潔に定義したりメタ関数をラップしたりする用途がある.
実際に使用される場合はこれらの用途が主である.
\lstinputlisting[caption=using alias templates to prevent specialization,label=cd:gen16]{code/generic/16.cpp}
%%%%%%%%%%%%%%%%%%%%%%%%%%%%%%%%%%%%%%%%%%%%%%%%%%%%%%
\subsection{オーバーロード}

%%%%%%%%%%%%%%%%%%%%%%%%%%%%%%%%%%%%%%%%%%%%%%%%%%%%%%
\subsubsection{オーバーロードの順番}

%%%%%%%%%%%%%%%%%%%%%%%%%%%%%%%%%%%%%%%%%%%%%%%%%%%%%%
\subsubsection{特殊化}

%%%%%%%%%%%%%%%%%%%%%%%%%%%%%%%%%%%%%%%%%%%%%%%%%%%%%%
\subsubsection{部分特殊化}
